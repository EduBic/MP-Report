
\section{Introduction}
\label{sec:introduction}

% Scrivere da dove nasce l’idea di fare Sweetie
% Per onor di causa citiamo Federico Allegro e l’origine dell’idea e del team

We design and develop Sweetie for create a first prototype in order to show a minimal product born from the idea to potential investors.
Once the users are coupled, they could share almost everything, text messages, images, photos, to-do lists send localized message. The main features built are:

\begin{itemize}
	\item Multiple chats based on the today's standard of messaging apps leader, WhatsApp and Telegram;
	\item Galleries that are containers of images grouped by both partners;
	\item Infos of Chat, Gallery and To-Do list, accessible from the menu of each, show the date of creation and, optional, the photo cover. Only for the galleries there is the optional placement, selectable through a Location Picker UI;
	\item To-Do lists managed by both, for any kind of topic, like grocery list, movies to watch, trips to plan;
	\item Geogift, are messages like a post-it, pictures or emoticons, geolocalized. The partner won't know about the gift until he/she will come exactly on that physical place e.g., a specific street address or store in the city. This thanks to the geofence, a virtual perimeter set on a real geographic area that should be monitored by a service. The simplest geofence to be created needs of center position, explicated by latitude and longitude, and the radius of circle area, setted in meters. When the boundary of geofence is crossed, the user is alerted by a push notification;
	\item Maps to see, as a markers, the Geogifts sent or discovered and the galleries that have a photo cover and a location specified;
	\item Calendar where to easily visualize the own saved message (bookmarked).
\end{itemize}

For that we don't put so much effort into usability and accessibility of an Android app, for example our app lacks of a initial tutorial or a welcome page where the user is instructed with what he could do with Sweetie furthermore some important feature: image compression and a complete notifications service were not implemented, we preferred to invest the last works time to test our application in order to fix underhand bugs.

The paper is organized as follows: section~\ref{sec:app-domain} describes how we analysis the domain of Sweetie and what are the entities that we manipulated for create the core business logic of the application. Section~\ref{sec:app-architecture} expose the app architecture, with merits and defects, and why we have done some choices. Section~\ref{sec:ext_api} show the list of external library that are used in order to support some feature of Sweetie. In section~\ref{sec:issues} we group all the major problematics that requires more effort and we explain the solution or workaround that we implements with some accepted trade-offs. Section~\ref{sec:futureworks}  show some important feature that our app has not yet implemented. Finally the section ~\ref{sec:conclusion} describes our considerations on the development of this project.
